\documentclass[a4paper,12pt]{article}
\usepackage[utf8]{vietnam}
\usepackage{geometry}
\geometry{a4paper, margin=1in}
\usepackage{booktabs}
\usepackage{longtable}
\usepackage{array}
\usepackage{amsmath}
\usepackage{hyperref}

% Defining packages for tables and formatting
\usepackage{caption}
\usepackage{colortbl}
\usepackage{xcolor}

% Font configuration (using Noto Serif as recommended)
\usepackage{fontspec} % Note: This is commented out as per guidelines to avoid fontspec with PDFLaTeX
%\setmainfont{Noto Serif} % Uncomment only if using XeLaTeX for non-Latin scripts

% Preamble setup
\usepackage{graphicx}
\usepackage{fancyhdr}
\pagestyle{fancy}
\fancyhf{}
\fancyhead[L]{Test Report - SecureInsure-Testing}
\fancyhead[R]{Page \thepage}
\fancyfoot[C]{\today}

\begin{document}

\title{Test Report - SecureInsure-Testing Project}
\author{Group Project (2 Members) \\ Supervisor: MSc. Võ Ngọc Tấn Phước}
\date{\today}
\maketitle

\section*{Introduction}
Generating a comprehensive test report detailing the evaluation of the SecureInsure-Testing project, focusing on API security and cross-platform application development.

\section{Resource Requirements for Testing}
Defining hardware and software needed for executing test cases.

\subsection{Yêu cầu tài nguyên kiểm thử}
\begin{itemize}
    \item \textbf{Phần cứng}: PC có internet.
    \item \textbf{Phần mềm}: XAMPP, Chrome, Postman, Java JDK, Maven.
\end{itemize}

\section{Test Case Overview}
Presenting the list of test scenarios executed during the project.

\subsection{Danh sách các tình huống kiểm thử}
\begin{longtable}{|c|p{2cm}|p{4cm}|p{7cm}|}
    \hline
    \textbf{STT} & \textbf{Mã testcase} & \textbf{Tên testcase} & \textbf{Mô tả} \\ \hline
    \endhead
    \hline
    \multicolumn{4}{|r|}{\textit{Continued on next page}} \\
    \hline
    \endfoot
    \hline
    \endlastfoot
    1 & TC_CUSTOMER_WEB_LOGIN_01 & Đăng nhập thành công với tài khoản Customer tồn tại qua /login & Từ trang chủ, chọn "đăng nhập", nhập email và mật khẩu hợp lệ, kiểm tra thông báo thành công. \\ \hline
    2 & TC_CUSTOMER_WEB_LOGIN_02 & Đăng nhập thất bại với tài khoản Customer bị chặn qua /login & Từ trang chủ, chọn "đăng nhập", không nhập email, kiểm tra thông báo lỗi. \\ \hline
    3 & TC_ADMIN_WEB_ADDUSER_01 & Thêm Admin mới thành công & Thêm user mới, kiểm tra giao diện và database, dự kiến lỗi. \\ \hline
    4 & TC_ADMIN_WEB_ADDUSER_02 & Thêm Admin với email đã tồn tại & Thêm user mới, kiểm tra giao diện và database, dự kiến lỗi. \\ \hline
    5 & TC_ADMIN_WEB_EDITUSER_01 & Sửa thông tin Khách hàng thành công & Sửa thông tin user, kiểm tra database, giao diện hiển thị, đảm bảo thành công. \\ \hline
    6 & TC_ADMIN_WEB_DELETEUSER_01 & Xóa Khách hàng thành công & Xóa user, kiểm tra giao diện và database, đảm bảo thành công. \\ \hline
    7 & TC_LOGIN_APP_01 & Đăng nhập qua App với user không tồn tại (Lần 1) & Đăng nhập với email không tồn tại, kiểm tra response và giao diện lỗi. \\ \hline
    8 & TC_LOGIN_APP_02 & Đăng nhập qua App với user không tồn tại (Lần 2) & Đăng nhập với email không tồn tại, kiểm tra response và thông báo lỗi trên giao diện. \\ \hline
    9 & TC_LOGIN_APP_03 & Đăng nhập qua App với user admin bị chặn (Lần 1) & Đăng nhập với user admin có status "Blocked", kiểm tra response và giao diện lỗi. \\ \hline
    10 & TC_LOGIN_APP_04 & Đăng nhập qua App với user admin bị chặn (Lần 2) & Đăng nhập với user admin có status "Blocked", kiểm tra response và thông báo lỗi. \\ \hline
    11 & TC_ROLE_APP_01 & Phân quyền truy cập TaskPanel (Lần 1) & Kiểm tra phân quyền: admin vào TaskPanel, customer bị chặn và chuyển hướng. \\ \hline
    12 & TC_ROLE_APP_02 & Phân quyền truy cập TaskPanel (Lần 2) & Kiểm tra phân quyền với logic bổ sung, kiểm tra lỗi NullPointerException. \\ \hline
    13 & TC_ROLE_APP_03 & Phân quyền truy cập TaskPanel (Lần 3) & Kiểm tra phân quyền sau khi sửa logic, đảm bảo customer bị chặn đúng cách. \\ \hline
    14 & TC_GETUSER_APP_01 & Hiển thị danh sách người dùng trên TaskPanel (Lần 1) & Kiểm tra hiển thị danh sách người dùng, kiểm tra nút refresh không làm mất dữ liệu. \\ \hline
    15 & TC_GETUSER_APP_02 & Hiển thị danh sách người dùng trên TaskPanel (Lần 2) & Kiểm tra hiển thị danh sách, một số trường bị thiếu (null), kiểm tra refresh. \\ \hline
    16 & TC_GETUSER_APP_03 & Hiển thị danh sách người dùng trên TaskPanel (Lần 3) & Kiểm tra hiển thị danh sách người dùng đầy đủ, kiểm tra refresh giữ nguyên dữ liệu. \\ \hline
    17 & TC_ADDUSER_APP_01 & Thêm người dùng mới trên TaskPanel (Lần 1) & Thêm user mới, kiểm tra giao diện và database, dự kiến lỗi. \\ \hline
    18 & TC_ADDUSER_APP_02 & Thêm người dùng mới trên TaskPanel (Lần 2) & Thêm user mới, kiểm tra giao diện và database, dự kiến lỗi 404. \\ \hline
    19 & TC_ADDUSER_APP_03 & Thêm người dùng mới trên TaskPanel (Lần 3) & Thêm user mới, kiểm tra giao diện và database, đảm bảo thành công. \\ \hline
    20 & TC_ADDUSER_APP_04 & Thêm người dùng với email đã tồn tại (Lần 1) & Thêm user với email đã tồn tại, kiểm tra thông báo lỗi và database không cập nhật. \\ \hline
    21 & TC_ADDUSER_APP_04 & Thêm người dùng với email đã tồn tại (Lần 2) & Thêm user với email đã tồn tại, kiểm tra thông báo lỗi và database không cập nhật. \\ \hline
    22 & TC_DELETEUSER_APP_01 & Xóa người dùng trên TaskPanel (Lần 1) & Xóa user, kiểm tra giao diện và database, dự kiến lỗi. \\ \hline
    23 & TC_DELETEUSER_APP_02 & Xóa người dùng trên TaskPanel (Lần 2) & Xóa user, kiểm tra giao diện và database, dự kiến lỗi. \\ \hline
    24 & TC_DELETEUSER_APP_03 & Xóa người dùng trên TaskPanel (Lần 3) & Xóa user, kiểm tra giao diện và database, đảm bảo thành công. \\ \hline
    25 & TC_UPDATEUSER_APP_01 & Sửa thông tin người dùng trên TaskPanel (Lần 1) & Sửa thông tin user, kiểm tra EditUserDialog và database, dự kiến lỗi. \\ \hline
    26 & TC_UPDATEUSER_APP_02 & Sửa thông tin người dùng trên TaskPanel (Lần 2) & Sửa thông tin user, kiểm tra EditUserDialog và database, dự kiến lỗi 500. \\ \hline
    27 & TC_UPDATEUSER_APP_03 & Sửa thông tin người dùng trên TaskPanel (Lần 3) & Sửa thông tin user, kiểm tra EditUserDialog và database, đảm bảo thành công. \\ \hline
    28 & TC_LOGOUT_APP_01 & Đăng xuất và quay lại màn hình đăng nhập (Lần 1) & Đăng xuất, kiểm tra giao diện và token, dự kiến lỗi (ứng dụng đóng). \\ \hline
    29 & TC_LOGOUT_APP_02 & Đăng xuất và quay lại màn hình đăng nhập (Lần 2) & Đăng xuất, kiểm tra giao diện và token, dự kiến lỗi (token không xóa). \\ \hline
    30 & TC_LOGOUT_APP_03 & Đăng xuất và quay lại màn hình đăng nhập (Lần 3) & Đăng xuất, kiểm tra giao diện và token, đảm bảo thành công. \\ \hline
    31 & TC_Token_Stored_01 & Kiểm tra token trong thư mục tokens được mã hóa & Kiểm tra token lưu trữ sau đăng nhập, đảm bảo mã hóa đúng (Base64). \\ \hline
    32 & TC_API_Secu_01 & Kiểm tra yêu cầu CSRF token trên endpoint /add & Gửi POST request tới /add không kèm CSRF token, kiểm tra response và database. \\ \hline
    33 & TC_API_Secu_02 & Kiểm tra POST request tới /add với CSRF token hợp lệ & Gửi POST request tới /add với CSRF token hợp lệ, kiểm tra response và database. \\ \hline
    34 & TC_API_Secu_03 & Kiểm tra chống SQL Injection trên /AdminUpdate & Gửi request với dữ liệu chứa mã SQL injection, kiểm tra response và database. \\ \hline
\end{longtable}

\section{Test Case Specifications}
Detailing the specific test cases for different actors and platforms.

\subsection{5.3.1. Test Case Actor Customer (Web Platform)}
\begin{longtable}{|p{2cm}|p{3cm}|p{3cm}|p{4cm}|p{3cm}|p{3cm}|p{3cm}|p{2cm}|}
    \hline
    \textbf{ID trường hợp thử nghiệm} & \textbf{Mô tả trường hợp thử nghiệm} & \textbf{Tiền điều kiện} & \textbf{Các bước kiểm tra} & \textbf{Dữ liệu thử nghiệm} & \textbf{Kết quả mong đợi} & \textbf{Kết quả thực tế} & \textbf{Vượt qua/thất bại (P/F)} \\ \hline
    \endhead
    \hline
    \multicolumn{8}{|r|}{\textit{Continued on next page}} \\
    \hline
    \endfoot
    \hline
    \endlastfoot
    TC_CUSTOMER_WEB_LOGIN_01 & Đăng nhập thành công với tài khoản Customer tồn tại qua /login & - Tài khoản customer1@gmail.com tồn tại trong bảng account với GoogleID: 12345, role: customer, status: Active. <br> - Server API đang chạy tại http://localhost/API_Secu & 1. Truy cập trang chủ web (http://localhost:80/API_Security/design/pages/trangchu). <br> 2. Nhấn nút "Login with Google". <br> 3. Nhập customer1@gmail.com và cấp quyền qua Google OAuth. <br> 4. Gửi yêu cầu POST đến /login với CSRF token và Google ID token. <br> 5. Kiểm tra phản hồi API và giao diện. & Request: POST /app_login <br> Header: Authorization: Bearer <Google_ID_Token>, X-CSRF-Token: <valid_csrf_token> <br> Body: { "GoogleID": "12345", "email": "customer1@gmail.com", "FullName": "Nguyen Van A", "access_token": "valid_access_token", "expires_at": "2025-06-01 12:00:00" } & Response: HTTP 200 OK, JSON { "status": "success", "token": "<JWT_token>", "message": "Đăng nhập thành công", "role": "customer", "status": "Active" }. <br> Giao diện chuyển sang dashboard Customer với các chức năng xem/sửa thông tin, gói bảo hiểm, hợp đồng. <br> Thông báo: "Đăng nhập thành công". & Response: HTTP 200 OK, JSON { "status": "success", "token": "<JWT_token>", "message": "Đăng nhập thành công", "role": "customer", "status": "Active" }. <br> Dashboard Customer hiển thị đúng. <br> Thông báo: "Đăng nhập thành công". & P \\ \hline
    TC_CUSTOMER_WEB_LOGIN_02 & Đăng nhập thất bại với tài khoản Customer bị chặn qua /login & - Tài khoản blockedcustomer@gmail.com tồn tại với GoogleID: 67890, role: customer, status: Blocked. <br> - Server API đang chạy. & 1. Truy cập trang chủ web. <br> 2. Nhấn nút "Login with Google". <br> 3. Nhập blockedcustomer@gmail.com. <br> 4. Gửi yêu cầu POST đến /login với CSRF token. <br> 5. Kiểm tra phản hồi API và giao diện. & Request: POST /app_login <br> Header: Authorization: Bearer <Google_ID_Token>, X-CSRF-Token: <valid_csrf_token> <br> Body: { "GoogleID": "67890", "email": "blockedcustomer@gmail.com", "FullName": "Nguyen Van B", "access_token": "valid_access_token", "expires_at": "2025-06-01 12:00:00" } & Response: HTTP 200 OK, JSON { "status": "success", "token": "<JWT_token>", "message": "Đăng nhập thành công", "role": "customer", "status": "Blocked" }. <br> Giao diện hiển thị thông báo lỗi: "Tài khoản bị chặn". <br> Vẫn ở trang đăng nhập. & Response: HTTP 200 OK, JSON { "status": "success", "token": "<JWT_token>", "message": "Đăng nhập thành công", "role": "customer", "status": "Blocked" }. <br> Thông báo lỗi hiển thị đúng. <br> Giao diện không chuyển. & P \\ \hline
\end{longtable}

\subsection{5.3.2. Test Case Actor Admin (Web Platform)}
\begin{longtable}{|p{2cm}|p{3cm}|p{3cm}|p{4cm}|p{3cm}|p{3cm}|p{3cm}|p{2cm}|}
    \hline
    \textbf{ID trường hợp thử nghiệm} & \textbf{Mô tả trường hợp thử nghiệm} & \textbf{Tiền điều kiện} & \textbf{Các bước kiểm tra} & \textbf{Dữ liệu thử nghiệm} & \textbf{Kết quả mong đợi} & \textbf{Kết quả thực tế} & \textbf{Vượt qua/thất bại (P/F)} \\ \hline
    \endhead
    \hline
    \multicolumn{8}{|r|}{\textit{Continued on next page}} \\
    \hline
    \endfoot
    \hline
    \endlastfoot
    TC_ADMIN_WEB_ADDUSER_01 & Thêm Admin mới thành công & - Admin admin1@gmail.com đã đăng nhập. <br> - Server API đang chạy. & 1. Vào trang quản lý người dùng. <br> 2. Nhấn nút "Thêm Admin". <br> 3. Nhập thông tin người dùng mới. <br> 4. Nhấn "Lưu". <br> 5. Kiểm tra giao diện và bảng account, admin trong MySQL. & Email: newadmin@gmail.com <br> FullName: "Nguyen Van A" <br> Role: admin <br> Status: Active & Thông báo: "Thêm Admin thành công". <br> Bảng account và admin có bản ghi mới với email newadmin@gmail.com. <br> Danh sách người dùng trên giao diện cập nhật. & Thông báo: "Thêm Admin thành công". <br> Bảng account và admin có bản ghi mới. <br> Danh sách người dùng cập nhật đúng. & P \\ \hline
    TC_ADMIN_WEB_ADDUSER_02 & Thêm Admin với email đã tồn tại & - Admin admin1@gmail.com đã đăng nhập. <br> - Tài khoản existing@gmail.com đã tồn tại. <br> - Server API đang chạy. & 1. Vào trang quản lý người dùng. <br> 2. Nhấn nút "Thêm Admin". <br> 3. Nhập thông tin với email đã tồn tại. <br> 4. Nhấn "Lưu". <br> 5. Kiểm tra giao diện và cơ sở dữ liệu. & Email: existing@gmail.com <br> FullName: "Nguyen Van B" <br> Role: admin <br> Status: Active & Thông báo lỗi: "Email đã tồn tại". <br> Cơ sở dữ liệu không có bản ghi mới. & Thông báo lỗi: "Email này đã được sử dụng cho tài khoản khác". <br> Cơ sở dữ liệu không thay đổi. & P \\ \hline
    TC_ADMIN_WEB_EDITUSER_01 & Sửa thông tin Khách hàng thành công & - Admin admin1@gmail.com đã đăng nhập. <br> - Tài khoản Khách hàng customer1@gmail.com tồn tại. <br> - Server API đang chạy. & 1. Vào trang quản lý người dùng. <br> 2. Chọn Khách hàng customer1@gmail.com. <br> 3. Nhấn "Sửa". <br> 4. Cập nhật số điện thoại. <br> 5. Nhấn "Lưu". <br> 6. Kiểm tra giao diện và bảng account. & Email: customer1@gmail.com <br> Phone: "0987654321" & Thông báo: "Cập nhật thành công". <br> Bảng account cập nhật số điện thoại mới cho customer1@gmail.com. <br> Giao diện hiển thị thông tin đã cập nhật. & Thông báo: "Cập nhật thành công". <br> Bảng account cập nhật đúng số điện thoại. <br> Giao diện hiển thị thông tin mới. & P \\ \hline
    TC_ADMIN_WEB_DELETEUSER_01 & Xóa Khách hàng thành công & - Admin admin1@gmail.com đã đăng nhập. <br> - Tài khoản Khách hàng customer2@gmail.com tồn tại. <br> - Server API đang chạy. & 1. Vào trang quản lý người dùng. <br> 2. Chọn Khách hàng customer2@gmail.com. <br> 3. Nhấn "Xóa". <br> 4. Xác nhận xóa. <br> 5. Kiểm tra giao diện và bảng account, customer. & Email: customer2@gmail.com & Thông báo: "Xóa thành công". <br> Bản ghi customer2@gmail.com bị xóa khỏi bảng account và customer. <br> Danh sách người dùng cập nhật. & Thông báo: "Xóa thành công". <br> Bản ghi bị xóa đúng. <br> Danh sách người dùng cập nhật. & P \\ \hline
\end{longtable}

\subsection{5.3.3.1. Test Case Đăng nhập}
\begin{longtable}{|p{2cm}|p{3cm}|p{3cm}|p{4cm}|p{3cm}|p{3cm}|p{3cm}|p{2cm}|}
    \hline
    \textbf{ID trường hợp thử nghiệm} & \textbf{Mô tả trường hợp thử nghiệm} & \textbf{Tiền điều kiện} & \textbf{Các bước kiểm tra} & \textbf{Dữ liệu thử nghiệm} & \textbf{Kết quả mong đợi} & \textbf{Kết quả thực tế} & \textbf{Vượt qua/thất bại (P/F)} \\ \hline
    \endhead
    \hline
    \multicolumn{8}{|r|}{\textit{Continued on next page}} \\
    \hline
    \endfoot
    \hline
    \endlastfoot
    TC_LOGIN_APP_01 & Đăng nhập qua APP với user không tồn tại (Lần 1). & - User testuser18@gmail.com chưa tồn tại trong database. <br> - Server API (http://localhost/API_Secu) đang chạy. & 1. Chạy ứng dụng Java và đăng nhập với email chưa tồn tại trong cơ sở dữ liệu. <br> 2. Bấm nút “Login with google”. <br> 3. Kiểm tra response và giao diện. & Email: “camtu18@gmail.com” & - Response: {"error": "Tài khoản không tồn tại hoặc chưa được admin tạo."}. <br> - Giao diện hiển thị thông báo lỗi và không chuyển sang TaskPanel. & - Response: {"error": "403 Forbidden"}. <br> - Giao diện không hiển thị thông báo lỗi cụ thể, vẫn ở màn hình đăng nhập. & F \\ \hline
    TC_LOGIN_APP_02 & Đăng nhập qua APP với user không tồn tại (Lần 2). & - User camtu18@gmail.com chưa tồn tại trong database. <br> - Server API đang chạy. <br> - Đã thêm logic kiểm tra user trong ApiController.php. & 1. Chạy ứng dụng Java và đăng nhập với camtu18@gmail.com. <br> 2. Bấm nút “Login with google”. <br> 3. Kiểm tra response và giao diện. & Email: “camtu18@gmail.com” & - Response: {"error": "Tài khoản không tồn tại hoặc chưa được admin tạo."}. <br> - Giao diện hiển thị thông báo lỗi và không chuyển sang TaskPanel. & - Response: {"error": "Tài khoản không tồn tại hoặc chưa được admin tạo."}. <br> - Giao diện hiển thị thông báo: "Tài khoản không tồn tại hoặc chưa được admin tạo. Vui lòng truy cập ứng dụng web tại: “http://localhost:80/API_Security/design/pages/trangchu ", không chuyển sang TaskPanel. & P \\ \hline
    TC_LOGIN_APP_03 & Đăng nhập qua APP với user admin đã tồn tại nhưng đã bị chặn quyền (Lần 1). & - User camtu7092003@gmail.com được tạo với role: "admin", status: "Blocked". & 1. Chạy ứng dụng Java và đăng nhập với camtu7092003@gmail.com. <br> 2. Bấm nút “Login with google”. <br> 3. Kiểm tra response và giao diện. & Email: “camtu7092003@gmail.com” & - Response: {"error": "Bạn chưa được cấp quyền để truy cập ứng dụng này. Tài khoản của bạn (camtu7092003@gmail.com) đã bị chặn."}. <br> - Giao diện hiển thị thông báo lỗi và không chuyển sang TaskPanel. & - Người dùng vẫn có thể truy cập được vào ứng dụng. & F \\ \hline
    TC_LOGIN_APP_04 & Đăng nhập qua APP với user admin đã tồn tại nhưng đã bị chặn quyền (Lần 2). & - User camtu7092003@gmail.com được tạo với role: "admin", status: "Blocked". & 1. Chạy ứng dụng Java và đăng nhập với camtu7092003@gmail.com. <br> 2. Bấm nút “Login with google”. <br> 3. Kiểm tra response và giao diện. & Email: “camtu7092003@gmail.com” & - Response: {"error": "Bạn chưa được cấp quyền để truy cập ứng dụng này. Tài khoản của bạn (camtu7092003@gmail.com) đã bị chặn."}. <br> - Giao diện hiển thị thông báo lỗi và không chuyển sang TaskPanel. & - Response: {"error": "Bạn chưa được cấp quyền để truy cập ứng dụng này. Tài khoản của bạn (camtu7092003@gmail.com) đã bị chặn."}. <br> - Giao diện hiển thị thông báo lỗi và không chuyển sang TaskPanel. & P \\ \hline
\end{longtable}

\subsection{5.3.3.2. Test Case Kiểm tra phân quyền truy cập dựa trên vai trò}
\begin{longtable}{|p{2cm}|p{3cm}|p{3cm}|p{4cm}|p{3cm}|p{3cm}|p{3cm}|p{2cm}|}
    \hline
    \textbf{ID trường hợp thử nghiệm} & \textbf{Mô tả trường hợp thử nghiệm} & \textbf{Tiền điều kiện} & \textbf{Các bước kiểm tra} & \textbf{Dữ liệu thử nghiệm} & \textbf{Kết quả mong đợi} & \textbf{Kết quả thực tế} & \textbf{Vượt qua/thất bại (P/F)} \\ \hline
    \endhead
    \hline
    \multicolumn{8}{|r|}{\textit{Continued on next page}} \\
    \hline
    \endfoot
    \hline
    \endlastfoot
    TC_ROLE_APP_01 & Phân quyền truy cập TaskPanel dựa trên vai trò (Lần 1). & - User camtu7092003@gmail.com được tạo với role: "admin", status: "Active". <br> - User tuchuc848@gmail.com được tạo với role: "customer", status: "Active". & 1. Đăng nhập với camtu7092003@gmail.com trên app. <br> 2. Kiểm tra giao diện. <br> 3. Đăng xuất, sau đó đăng nhập với tuchuc848@gmail.com. <br> 4. Kiểm tra giao diện. & - Email 1: “camtu18@gmail.com” <br> - Email 2: tuchuc848@gmail.com & - Với camtu18@gmail.com, giao diện chuyển sang TaskPanel. <br> - Với tuchuc848@gmail.com, hiển thị thông báo lỗi và yêu cầu chuyển hướng đến web (http://localhost:80/API_Security/design/pages/trangchu). & - Với testuser15@gmail.com, giao diện chuyển sang TaskPanel. <br> - Với testuser16@gmail.com, giao diện vẫn chuyển sang TaskPanel (không đúng). & F \\ \hline
    TC_ROLE_APP_02 & Phân quyền truy cập TaskPanel dựa trên vai trò (Lần 2) & - User camtu7092003@gmail.com được tạo với role: "admin", status: "Active". <br> - User tuchuc848@gmail.com được tạo với role: "customer", status: "Active". <br> - Đã thêm kiểm tra authService.getLastLoginRole() trong MainWindow.java. <br> - Đã thêm logic kiểm tra user trong ApiController.php. & 1. Đăng nhập với camtu7092003@gmail.com trên app. <br> 2. Kiểm tra giao diện. <br> 3. Đăng xuất, sau đó đăng nhập với tuchuc848@gmail.com. <br> 4. Kiểm tra giao diện. & - Email 1: “camtu18@gmail.com” <br> - Email 2: tuchuc848@gmail.com & - Với camtu18@gmail.com, giao diện chuyển sang TaskPanel. <br> - Với tuchuc848@gmail.com, hiển thị thông báo lỗi và yêu cầu chuyển hướng đến web (http://localhost:80/API_Security/design/pages/trangchu). & - Với testuser15@gmail.com, lỗi NullPointerException xuất hiện trong log. <br> - Với testuser16@gmail.com, lỗi NullPointerException xuất hiện, không hiển thị thông báo. & F \\ \hline
    TC_ROLE_APP_03 & Phân quyền truy cập TaskPanel dựa trên vai trò (Lần 3) & - User camtu7092003@gmail.com được tạo với role: "admin", status: "Active". <br> - User tuchuc848@gmail.com được tạo với role: "customer", status: "Active". <br> - Đã sửa AuthService.extractRoleFromToken để gán đúng lastLoginRole. & 1. Đăng nhập với camtu7092003@gmail.com trên app. <br> 2. Kiểm tra giao diện. <br> 3. Đăng xuất, sau đó đăng nhập với tuchuc848@gmail.com. <br> 4. Kiểm tra giao diện. & - Email 1: “camtu18@gmail.com” <br> - Email 2: tuchuc848@gmail.com & - Với camtu18@gmail.com, giao diện chuyển sang TaskPanel. <br> - Với tuchuc848@gmail.com, hiển thị thông báo lỗi và yêu cầu chuyển hướng đến web (http://localhost:80/API_Security/design/pages/trangchu). & - Với testuser15@gmail.com, giao diện chuyển sang TaskPanel. <br> - Với testuser16@gmail.com, hiển thị thông báo: "Bạn chưa được cấp quyền để truy cập ứng dụng này. Vui lòng truy cập ứng dụng web tại: “http://localhost:80/API_Security/design/pages/trangchu ". & P \\ \hline
\end{longtable}

\subsection{5.3.3.3. Test Case Kiểm tra chức năng hiển thị thông tin người dùng}
\begin{longtable}{|p{2cm}|p{3cm}|p{3cm}|p{4cm}|p{3cm}|p{3cm}|p{3cm}|p{2cm}|}
    \hline
    \textbf{ID trường hợp thử nghiệm} & \textbf{Mô tả trường hợp thử nghiệm} & \textbf{Tiền điều kiện} & \textbf{Các bước kiểm tra} & \textbf{Dữ liệu thử nghiệm} & \textbf{Kết quả mong đợi} & \textbf{Kết quả thực tế} & \textbf{Vượt qua/thất bại (P/F)} \\ \hline
    \endhead
    \hline
    \multicolumn{8}{|r|}{\textit{Continued on next page}} \\
    \hline
    \endfoot
    \hline
    \endlastfoot
    TC_GETUSER_APP_01 & Hiển thị danh sách người dùng trên TaskPanel (Lần 1). & - User testuser15@gmail.com đã đăng nhập với role: "admin". <br> - Có tồn tại dữ liệu về người dùng trong cơ sở dữ liệu. <br> - Server API đang chạy & 1. Kiểm tra Table hiển thị danh sách công việc. <br> 2. Thử bấm nút refresh giao diện. & - & - Table hiển thị đúng các dữ liệu về thông tin người dùng từ cơ sở dữ liệu. <br> - Sau khi refresh, dữ liệu vẫn hiển thị, không biến mất. & - Table không hiển thị danh sách về thông tin người dùng như trong cơ sở dữ liệu. <br> - Nhấn nút Refresh cũng không load được thông tin người dùng. & F \\ \hline
    TC_GETUSER_APP_02 & Hiển thị danh sách người dùng trên TaskPanel (Lần 2). & - User testuser15@gmail.com đã đăng nhập với role: "admin". <br> - Có tồn tại dữ liệu về người dùng trong cơ sở dữ liệu. <br> - Server API đang chạy & 1. Kiểm tra Table hiển thị danh sách công việc. <br> 2. Thử bấm nút refresh giao diện. & - & - Table hiển thị đúng các dữ liệu về thông tin người dùng từ cơ sở dữ liệu. <br> - Sau khi refresh, dữ liệu vẫn hiển thị, không biến mất. & - Có hiển thị thông tin người dùng, nhưng một số trường bị thiếu (null). <br> - Nhấn nút Refresh thông tin người dùng không bị biến mất. & F \\ \hline
    TC_GETUSER_APP_03 & Hiển thị danh sách người dùng trên TaskPanel (Lần 3). & - User testuser15@gmail.com đã đăng nhập với role: "admin". <br> - Có tồn tại dữ liệu về người dùng trong cơ sở dữ liệu. <br> - Server API đang chạy & 1. Kiểm tra Table hiển thị danh sách công việc. <br> 2. Thử bấm nút refresh giao diện. & - & - Table hiển thị đúng các dữ liệu về thông tin người dùng từ cơ sở dữ liệu. <br> - Sau khi refresh, dữ liệu vẫn hiển thị, không biến mất. & - Table hiển thị đúng các dữ liệu về thông tin người dùng từ cơ sở dữ liệu. <br> - Sau khi refresh, dữ liệu vẫn hiển thị, không biến mất. & P \\ \hline
\end{longtable}

\subsection{5.3.3.4. Test Case Kiểm tra chức năng thêm thông tin người dùng}
\begin{longtable}{|p{2cm}|p{3cm}|p{3cm}|p{4cm}|p{3cm}|p{3cm}|p{3cm}|p{2cm}|}
    \hline
    \textbf{ID trường hợp thử nghiệm} & \textbf{Mô tả trường hợp thử nghiệm} & \textbf{Tiền điều kiện} & \textbf{Các bước kiểm tra} & \textbf{Dữ liệu thử nghiệm} & \textbf{Kết quả mong đợi} & \textbf{Kết quả thực tế} & \textbf{Vượt qua/thất bại (P/F)} \\ \hline
    \endhead
    \hline
    \multicolumn{8}{|r|}{\textit{Continued on next page}} \\
    \hline
    \endfoot
    \hline
    \endlastfoot
    TC_ADDUSER_APP_01 & Thêm người dùng mới trên TaskPanel (Lần 1). & - User testuser15@gmail.com đã đăng nhập với role: "admin". <br> - Server API đang chạy & 1. Nhấn nút "Add User", nhập thông tin người dùng. <br> 2. Kiểm tra giao diện và bảng account trong MySQL. & - email: testuser21@gmail.com <br> - họ tên: Cam Tu Nguyen <br> - sdt: 0909090 & - Người dùng mới xuất hiện trong Table. <br> - Bảng account chứa bản ghi mới. & - Thông báo lỗi: “Lỗi khi thêm người dùng!” <br> - Người dùng mới không xuất hiện trong JTable. <br> - Bảng account không có bản ghi mới. & F \\ \hline
    TC_ADDUSER_APP_02 & Thêm người dùng mới trên TaskPanel (Lần 2). & - User testuser15@gmail.com đã đăng nhập với role: "admin". <br> - Server API đang chạy & 1. Nhấn nút "Add User", nhập thông tin người dùng. <br> 2. Kiểm tra giao diện và bảng account trong MySQL. & - email: testuser21@gmail.com <br> - họ tên: Cam Tu Nguyen <br> - sdt: 0909090 & - Người dùng mới xuất hiện trong Table. <br> - Bảng account chứa bản ghi mới. & - Thông báo lỗi: “Lỗi khi thêm người dùng!” <br> - Người dùng mới không xuất hiện trong JTable. <br> - Bảng account không có bản ghi mới. <br> - Log: "404 Not Found - Endpoint /create not defined". & F \\ \hline
    TC_ADDUSER_APP_03 & Thêm người dùng mới trên TaskPanel (Lần 2). & - User testuser15@gmail.com đã đăng nhập với role: "admin". <br> - Server API đang chạy & 1. Nhấn nút "Add User", nhập thông tin người dùng. <br> 2. Kiểm tra giao diện và bảng account trong MySQL. & - email: testuser21@gmail.com <br> - họ tên: Cam Tu Nguyen <br> - sdt: 0909090 & - Người dùng mới xuất hiện trong Table. <br> - Bảng account chứa bản ghi mới. & - Người dùng mới xuất hiện trong Table. <br> - Bảng account có bản ghi mới. & P \\ \hline
    TC_ADDUSER_APP_04 & Thêm người dùng mới trên TaskPanel với email đã tồn tại trong cơ sở dữ liệu (Lần 1). & - User testuser15@gmail.com đã đăng nhập với role: "admin". <br> - Server API đang chạy <br> - Trong cơ sở dữ liệu có tồn tại user có email testuser22@gmail.com & 1. Nhấn nút "Add User", nhập thông tin người dùng. <br> 2. Kiểm tra giao diện và bảng account trong MySQL. & - email: testuser22@gmail.com <br> - họ tên: Cam Tu Nguyen <br> - sdt: 0909090 & - Thông báo lỗi: “Email này đã được sử dụng cho tài khoản khác. Vui lòng chọn email khác!” <br> - Trong Table và Database không cập nhật thêm user mới. & - Thông báo: “Thêm người dùng thành công!” <br> - Người dùng mới được cập nhật vào Table và Database. <br> - Trong cơ sở dữ liệu tồn tại 2 tài khoản có email giống nhau. & F \\ \hline
    TC_ADDUSER_APP_04 & Thêm người dùng mới trên TaskPanel với email đã tồn tại trong cơ sở dữ liệu (Lần 2). & - User testuser22@gmail.com đã đăng nhập với role: "admin". <br> - Server API đang chạy & 1. Nhấn nút "Add User", nhập thông tin người dùng. <br> 2. Kiểm tra giao diện và bảng account trong MySQL. & - email: testuser22@gmail.com <br> - họ tên: Cam Tu Nguyen <br> - sdt: 0909090 & - Thông báo lỗi: “Email này đã được sử dụng cho tài khoản khác. Vui lòng chọn email khác!” <br> - Trong Table và Database không cập nhật thêm user mới. & - Thông báo lỗi: “Email này đã được sử dụng cho tài khoản khác. Vui lòng chọn email khác!” <br> - Trong Table và Database không cập nhật thêm user mới. & P \\ \hline
\end{longtable}

\subsection{5.3.3.5. Test Case Kiểm tra chức năng xóa người dùng}
\begin{longtable}{|p{2cm}|p{3cm}|p{3cm}|p{4cm}|p{3cm}|p{3cm}|p{3cm}|p{2cm}|}
    \hline
    \textbf{ID trường hợp thử nghiệm} & \textbf{Mô tả trường hợp thử nghiệm} & \textbf{Tiền điều kiện} & \textbf{Các bước kiểm tra} & \textbf{Dữ liệu thử nghiệm} & \textbf{Kết quả mong đợi} & \textbf{Kết quả thực tế} & \textbf{Vượt qua/thất bại (P/F)} \\ \hline
    \endhead
    \hline
    \multicolumn{8}{|r|}{\textit{Continued on next page}} \\
    \hline
    \endfoot
    \hline
    \endlastfoot
    TC_DELETEUSER_APP_01 & Xóa người dùng trên TaskPanel (Lần 1). & - User testuser15@gmail.com đã đăng nhập với role: "admin". <br> - Server API đang chạy & 1. Chọn hàng dữ liệu của user muốn xóa. <br> 2. Nhấn nút “Delete user”. <br> 3. Nhấn “yes” trong bảng Confirm Delete. <br> 4. Kiểm tra giao diện và bảng account. & Hàng dữ liệu của tài khoản có mail testuser21@gmail.com & - Tài khoản và tất cả thông tin của testuser21@gmail.com bị xóa khỏi Table và database. & - Thông báo lỗi: “Lỗi khi xóa người dùng!” <br> - Tài khoản và tất cả thông tin của testuser21@gmail.com không bị xóa. <br> - Bảng account vẫn còn hiển thị người dùng testuser21@gmail.com. & F \\ \hline
    TC_DELETEUSER_APP_02 & Xóa người dùng trên TaskPanel (Lần 2). & - User testuser15@gmail.com đã đăng nhập với role: "admin". <br> - Server API đang chạy & 1. Chọn hàng dữ liệu của user muốn xóa. <br> 2. Nhấn nút “Delete user”. <br> 3. Nhấn “yes” trong bảng Confirm Delete. <br> 4. Kiểm tra giao diện và bảng account. & Hàng dữ liệu của tài khoản có mail testuser21@gmail.com & - Tài khoản và tất cả thông tin của testuser21@gmail.com bị xóa khỏi Table và database. & - Thông báo lỗi: “Lỗi khi xóa người dùng!” <br> - Tài khoản và tất cả thông tin của testuser21@gmail.com không bị xóa. <br> - Bảng account vẫn còn ghi của testuser21@gmail.com. & F \\ \hline
    TC_DELETEUSER_APP_03 & Xóa người dùng trên TaskPanel (Lần 3). & - User testuser15@gmail.com đã đăng nhập với role: "admin". <br> - Server API đang chạy & 1. Chọn hàng dữ liệu của user muốn xóa. <br> 2. Nhấn nút “Delete user”. <br> 3. Nhấn “yes” trong bảng Confirm Delete. <br> 4. Kiểm tra giao diện và bảng account. & Hàng dữ liệu của tài khoản có mail testuser21@gmail.com & - Tài khoản và tất cả thông tin của testuser21@gmail.com bị xóa khỏi Table và database. & - Tài khoản và tất cả thông tin của testuser21@gmail.com bị xóa khỏi Table. <br> - Bảng account không còn bản ghi testuser21@gmail.com. & P \\ \hline
\end{longtable}

\subsection{5.3.3.6. Test Case Kiểm tra chức năng sửa thông tin người dùng}
\begin{longtable}{|p{2cm}|p{3cm}|p{3cm}|p{4cm}|p{3cm}|p{3cm}|p{3cm}|p{2cm}|}
    \hline
    \textbf{ID trường hợp thử nghiệm} & \textbf{Mô tả trường hợp thử nghiệm} & \textbf{Tiền điều kiện} & \textbf{Các bước kiểm tra} & \textbf{Dữ liệu thử nghiệm} & \textbf{Kết quả mong đợi} & \textbf{Kết quả thực tế} & \textbf{Vượt qua/thất bại (P/F)} \\ \hline
    \endhead
    \hline
    \multicolumn{8}{|r|}{\textit{Continued on next page}} \\
    \hline
    \endfoot
    \hline
    \endlastfoot
    TC_UPDATEUSER_APP_01 & Sửa thông tin người dùng trên TaskPanel (Lần 1). & - User testuser15@gmail.com đã đăng nhập với role: "admin". <br> - Có người dùng testuser21@gmail.com trong database, các thông tin của người dùng được hiển thị. <br> - Server API đang chạy & 1. Chọn hàng của testuser21@gmail.com trong Table. <br> 2. Nhấn nút "Edit User". <br> 3. Kiểm tra EditUserDialog có hiện không. <br> 4. Thay đổi số điện thoại từ “0909” thành “7878”, nhấn "Save". <br> 5. Kiểm tra giao diện và database. & phone: “7878” & - EditUserDialog hiển thị thông tin hiện tại của user: email: "testuser21@gmail.com", role: "customer", status: "Active", phone: “0909”. <br> - Sau khi nhấn "Save", Table và database cập nhật phone: “7878”. & - EditUserDialog mở nhưng các trường trống (email, role, status không được điền). <br> - Nhấn "Save" không có tác dụng, Table và database không thay đổi. <br> - Log: "No data retrieved for selected row, invalid row selection in JTable". & F \\ \hline
    TC_UPDATEUSER_APP_02 & Sửa thông tin người dùng trên TaskPanel (Lần 2). & - User testuser15@gmail.com đã đăng nhập với role: "admin". <br> - Có người dùng testuser21@gmail.com trong database, các thông tin của người dùng được hiển thị. <br> - Server API đang chạy. <br> - Đã sửa logic lấy dữ liệu từ JTable bằng getSelectedRow() và getValueAt(). & 1. Chọn hàng của testuser21@gmail.com trong Table. <br> 2. Nhấn nút "Edit User". <br> 3. Kiểm tra EditUserDialog có hiện không. <br> 4. Thay đổi số điện thoại từ “0909” thành “7878”, nhấn "Save". <br> 5. Kiểm tra giao diện và database. & phone: “7878” & - EditUserDialog hiển thị thông tin hiện tại của user: email: "testuser21@gmail.com", role: "customer", status: "Active", phone: “0909”. <br> - Sau khi nhấn "Save", Table và database cập nhật phone: “7878”. & - EditUserDialog hiển thị đúng thông tin: email: "testuser21@gmail.com", role: "customer", status: "Active", phone: “7878”. <br> - Sau khi nhấn "Save", giao diện không cập nhật. <br> - Log: "500 Internal Server Error - Endpoint /update failed to process PUT request". <br> - Database không thay đổi. & F \\ \hline
    TC_UPDATEUSER_APP_03 & Sửa thông tin người dùng trên TaskPanel (Lần 3). & - User testuser15@gmail.com đã đăng nhập với role: "admin". <br> - Có người dùng testuser21@gmail.com trong database, các thông tin của người dùng được hiển thị. <br> - Server API đang chạy. <br> - Đã thêm case 'update' trong ApiController.php với phương thức PUT. & 1. Chọn hàng của testuser21@gmail.com trong Table. <br> 2. Nhấn nút "Edit User". <br> 3. Kiểm tra EditUserDialog có hiện không. <br> 4. Thay đổi số điện thoại từ “0909” thành “7878”, nhấn "Save". <br> 5. Kiểm tra giao diện và database. & phone: “7878” & - EditUserDialog hiển thị thông tin hiện tại của user: email: "testuser21@gmail.com", role: "customer", status: "Active", phone: “0909”. <br> - Sau khi nhấn "Save", Table và database cập nhật phone: “7878”. & - EditUserDialog hiển thị đúng thông tin: email: "testuser21@gmail.com", role: "customer", status: "Active", phone: “7878”. <br> - Sau khi nhấn "Save", Table cập nhật: phone: "7878". <br> - Database cập nhật: email: "testuser21@gmail.com", role: "admin", status: "Active", phone: “7878”. & P \\ \hline
\end{longtable}

\subsection{5.3.3.7. Test Case Kiểm tra đăng xuất}
\begin{longtable}{|p{2cm}|p{3cm}|p{3cm}|p{4cm}|p{3cm}|p{3cm}|p{3cm}|p{2cm}|}
    \hline
    \textbf{ID trường hợp thử nghiệm} & \textbf{Mô tả trường hợp thử nghiệm} & \textbf{Tiền điều kiện} & \textbf{Các bước kiểm tra} & \textbf{Dữ liệu thử nghiệm} & \textbf{Kết quả mong đợi} & \textbf{Kết quả thực tế} & \textbf{Vượt qua/thất bại (P/F)} \\ \hline
    \endhead
    \hline
    \multicolumn{8}{|r|}{\textit{Continued on next page}} \\
    \hline
    \endfoot
    \hline
    \endlastfoot
    TC_LOGOUT_APP_01 & Logout và quay lại màn hình đăng nhập (Lần 1). & - User testuser15@gmail.com đã đăng nhập với role: "admin". <br> - Server API đang chạy <br> - Token đã được lưu trong Preferences và user_tokens. & 1. Nhấn nút "Logout" trên TaskPanel. <br> 2. Kiểm tra giao diện và bảng user_tokens. & - & - Giao diện chuyển về LoginPanel. <br> - Token bị xóa khỏi Preferences và user_tokens. & - Ứng dụng đóng hoàn toàn, không chuyển về LoginPanel. <br> - Token không bị xóa khỏi Preferences và user_tokens. & F \\ \hline
    TC_LOGOUT_APP_02 & Logout và quay lại màn hình đăng nhập (Lần 2). & - User testuser15@gmail.com đã đăng nhập với role: "admin". <br> - Server API đang chạy & 1. Nhấn nút "Logout" trên TaskPanel. <br> 2. Kiểm tra giao diện và bảng user_tokens. & - & - Giao diện chuyển về LoginPanel. <br> - Token bị xóa khỏi Preferences và user_tokens. & - Giao diện chuyển về LoginPanel. <br> - Token vẫn tồn tại trong Preferences và user_tokens. & F \\ \hline
    TC_LOGOUT_APP_03 & Logout và quay lại màn hình đăng nhập (Lần 3). & - User testuser15@gmail.com đã đăng nhập với role: "admin". <br> - Server API đang chạy & 1. Nhấn nút "Logout" trên TaskPanel. <br> 2. Kiểm tra giao diện và bảng user_tokens. & - & - Giao diện chuyển về LoginPanel. <br> - Token bị xóa khỏi Preferences và user_tokens. & - Giao diện chuyển về LoginPanel. <br> - Token bị xóa khỏi Preferences và user_tokens. & P \\ \hline
\end{longtable}

\subsection{5.3.4. Test Case Bảo mật API}
\begin{longtable}{|p{2cm}|p{3cm}|p{3cm}|p{4cm}|p{3cm}|p{3cm}|p{3cm}|p{2cm}|}
    \hline
    \textbf{ID trường hợp thử nghiệm} & \textbf{Mô tả trường hợp thử nghiệm} & \textbf{Tiền điều kiện} & \textbf{Các bước kiểm tra} & \textbf{Dữ liệu thử nghiệm} & \textbf{Kết quả mong đợi} & \textbf{Kết quả thực tế} & \textbf{Vượt qua/thất bại (P/F)} \\ \hline
    \endhead
    \hline
    \multicolumn{8}{|r|}{\textit{Continued on next page}} \\
    \hline
    \endfoot
    \hline
    \endlastfoot
    TC_Token_Stored_01 & Kiểm tra token trong thư mục tokens được mã hóa & - Đăng nhập bằng Google OAuth với email tuchuc848@gmail.com. <br> - Thư mục tokens đã được tạo trong TaskManagerApp/tokens. & 1. Đăng nhập bằng Google OAuth với email tuchuc848@gmail.com. <br> 2. Mở thư mục C:\Users\PC\OneDrive - Industrial University of HoChiMinh City\Documents\NetBeansProjects\TaskManagerApp\tokens. <br> 3. Kiểm tra nội dung tệp StoredCredential. & Nội dung tệp StoredCredential là chuỗi mã hóa (dạng Base64, không đọc được trực tiếp). & - Response: {"error": "403 Forbidden - CSRF token missing or invalid"}. <br> - Bảng account không có bản ghi mới. & - Response: {"error": "403 Forbidden - CSRF token missing or invalid"}. <br> - Bảng account không có bản ghi mới. <br> - Log: " Invalid CSRF token ". & P (do thiếu CSRF token, đây là hành vi bảo mật mong đợi) \\ \hline
    TC_API_Secu_01 & - Gửi request POST tới /add để thêm người dùng mới nhưng không kèm theo csrf token. <br> - Mục đích: Xác minh rằng endpoint /add yêu cầu CSRF token hợp lệ khi sử dụng phương thức POST để thêm người dùng. & - Token hợp lệ của admin (tuchuc34@gmail.com) đã được tạo. <br> - CSRF token chưa được gửi. <br> - Server API (http://localhost/API_Secu) đang chạy. & 1. Gửi request POST tới http://localhost/API_Secu/add với: <br> Header: Authorization: Bearer <auth_token>, Content-Type: application/json. <br> Body: {"email": "tuchuc34@gmail.com", "role": "customer", "status": "Active"}. <br> 4. Kiểm tra response và bảng account. & - email: tuchuc34@gmail.com <br> - token được cấp của user sau khi đăng nhập & - Response: {"error": "403 Forbidden - CSRF token missing or invalid"}. <br> - Bảng account không có bản ghi mới. & - Response: {"error": "403 Forbidden - CSRF token missing or invalid"}. <br> - Bảng account không có bản ghi mới. <br> - Log: " Invalid CSRF token ". & P (do thiếu CSRF token, đây là hành vi bảo mật mong đợi) \\ \hline
    TC_API_Secu_02 & Gửi request POST tới /add có kèm csrf token hợp lệ. & - Token hợp lệ của admin (tuchuc34@gmail.com) đã được tạo. <br> - CSRF token hợp lệ (csrf_token: "valid_token") được tạo từ server và gửi. <br> - Server API đang chạy. & 1. Gửi request POST tới http://localhost/API_Secu/add với: <br> Header: Authorization: Bearer <auth_token>, Content-Type: application/json, X-CSRF-Token: valid_token. <br> Body: {"email": "tuchuc34@gmail.com", "role": "customer", "status": "Active"}. <br> 2. Kiểm tra response và bảng account. & - Email: tuchuc34@gmail.com <br> - Auth Token <br> - Csrf token & - Response: {"status": "success", "message": "Thêm người dùng thành công"}. <br> - Bảng account chứa bản ghi mới: email: "tuchuc34@gmail.com", role: "customer", status: "Active". & - Response: {"status": "success", "message": "Thêm người dùng thành công"}. <br> - Bảng account chứa bản ghi: email: "tuchuc34@gmail.com", role: "customer", status: "Active". & P \\ \hline
    TC_API_Secu_03 & Kiểm tra chống SQL Injection trên /AdminUpdate & - User tuchuc848@gmail.com (admin) đã đăng nhập. <br> - Server API đang chạy. & 1. Gửi request POST tới http://localhost/API_Secu/AdminUpdate với dữ liệu đầu vào chứa mã độc: <br> Header: Authorization: Bearer <token>, Content-Type: application/json. <br> Body: {"email": "tuchuc848@gmail.com", "role": "' OR '1'='1", "status": "Active", "csrf_token": "<valid_csrf_token>"}. <br> 2. Kiểm tra response và database. & email: "tuchuc848@gmail.com", role: "' OR '1'='1", status: "Active", csrf_token: "<valid_csrf_token> & - Response: {"status": "success", "message": "Cập nhật thành công"}. <br> - Database không bị thay đổi bất thường (role không bị thay đổi thành giá trị bất hợp lệ). & - Response: {"status": "success", "message": "Cập nhật thành công"}. <br> - Database cập nhật đúng: role vẫn là "' OR '1'='1" (được xử lý như chuỗi, không phải mã SQL). & (Không ghi P/F, cần kiểm tra lại) \\ \hline
\end{longtable}

\section{Conclusion}
Summarizing the testing outcomes and identifying areas for improvement based on the results.

\end{document}